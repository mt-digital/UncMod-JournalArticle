% \documentclass[11pt,letterpaper]{article}
\documentclass[letterpaper,11.5pt]{scrartcl}
% \documentclass[11pt]{report}
% \documentclass{report}
% \documentclass{book}
\usepackage[bookmarks, hidelinks]{hyperref}
\usepackage[chatter]{rotating}
\usepackage{amssymb,amsmath}
\usepackage[title]{appendix}
% \usepackage{fullpage}
\usepackage{tabulary}
\usepackage{tabularx}
\usepackage{float}
% \usepackage[margin=0.50in]{geometry}
\usepackage[margin=1.00in]{geometry}

\usepackage{booktabs}
\usepackage{pslatex}
\usepackage{apacite}
\usepackage{caption}
\usepackage{subcaption}
\usepackage{pgfplots}
\usepackage{wrapfig}
\usepackage[english]{babel}
\usepackage{lmodern}
\usepackage{setspace}
\doublespace
% \usepackage{url}
\usepackage{bigfoot}
\usepackage[export]{adjustbox}
\setlength\intextsep{0pt}

% Colored comments 
\usepackage{color}
\definecolor{myorange}{RGB}{240, 96, 0}
\newcommand{\mt}[1]{{\textcolor{myorange} {({\tiny MT:} #1)}}}

\definecolor{myblue}{RGB}{30,144,255}
\newcommand{\jhj}[1]{{\textcolor{myblue} {({\tiny JHJ:} #1)}}}

\definecolor{mypurple}{RGB}{148,0,211}
\newcommand{\cm}[1]{{\textcolor{mypurple} {({\tiny CM:} #1)}}}

\definecolor{mygreen}{RGB}{26, 153, 51}
\newcommand{\ps}[1]{{\textcolor{mygreen} {({\tiny PS:} #1)}}}


\usepackage{graphicx}

\title{Uncertainty in social learning evolution}

\author{{}}

\begin{document}
\maketitle

\newcommand{\pisub}[1]{\pi_{\mathrm{#1}}}
\newcommand{\pilow}{\pisub{low}}
\newcommand{\pihigh}{\pisub{high}}
\newcommand{\piI}{\langle \pisub{I} \rangle}
\newcommand{\piS}{\langle \pisub{S} \rangle}

\newcommand{\meanvar}[1]{\langle #1 \rangle}
\newcommand{\meansl}{\meanvar{s}}
\newcommand{\meanpi}{\meanvar{\pi}}
\newcommand{\meansoc}{\meanvar{\pi_\mathrm{S}}}
\newcommand{\meanasoc}{\meanvar{\pi_\mathrm{A}}}
\newcommand{\meanT}{\meanvar{T}}

\begin{abstract}

Social learning is essential to survival. It is likely to evolve when it is more
efficient than asocial, trial-and-error learning. Theoretically, some uncertainty is
necessary for social learning to be necessary, however too much uncertainty makes
social information useless as information becomes outdated. This fact is empirically
supported across biology and human sciences. However, we lack a theoretical
framework predict the effects of specific classes and types of uncertainty on social
learning. Furthermore, existing models and experimental operationalizations of
uncertainty are ambiguously related, and models of uncertainty and social learning
tend to only consider a small number of sources of uncertainty and behavioral
choices.  Here we use evolutionary agent-based modeling to improve on these
shortcomings. We model a time-varying environment with a varying number
of possible behaviors agents could perform to acquire payoffs.
We show that ambiguous payoffs, larger possible decision sets, and shorter agent
lifespans interact to affect social learning in complex ways that make evolutionary
selection more or less certain about the optimal strategy. Our work consolidates
models of uncertainty and social learning and models the evolution of social
learning that could be used to guide human, non-human, and artificial life
towards optimal responses to existential threats and new opportunities.
\end{abstract}


Social learning is essential to human and other species' everyday life and survival.
It allows individuals to solve problems when acquiring information from others is
more efficient than learning on one's own~\cite{Laland2004}. Theory predicts that
social learning should be favored in contexts with greater
uncertainty~\cite{BoydRicherson1985,Henrich1998}, and this prediction has received
some empirical support across species~\cite{McElreath2005,Kendal2018,Allen2019}.
However, the meaning of the term \emph{uncertainty} is not always clear, and often
conflates environmental variability, spatial heterogeneity, ambiguity or
uncertainty about payoff structure, and other possible interpretations of
\emph{uncertainty}.  Moreover, most models of the do not account for
individual-level cognition~\cite{Heyes2016}. 


\end{document}
